\chapter{Relevant Literature Study} 
\label{sec:literature}

\ifpdf
    \graphicspath{{2_literature/figures/PNG/}{2_lirterature/figures/}}
\else
    \graphicspath{{2_literature/figures/EPS/}{2_literature/figures/}}
\fi

Particle identification and categorisation is important in particle physics. Common practice for characterising such particles includes reconstruction of clusters, tracks, jets, rings and showers associated with particle interactions \cite{aurisano2016convolutional}. Compared to several other domains that have seen significant adoption of deep learning, particle physics has remained relatively conservative in adopting deep learning models. This chapter summarises the few neutrino studies and experiments based on the type of network employed. Observations from these efforts serve as points of study for this thesis. 

\section{Feed-Forward Networks}
Szadkowski et al. (2014) proposed a three-layer Neural Network (NN) to perform pattern recognition and classify proton old showers and neutrino young showers. Amidst background noise of cosmic rays, detecting the very infrequently occurring neutrino showers has been the main challenge for the field. The NN was set up to identify both young and old showers using simulated Monte Carlo events. The network was trained on 245,760 different patterns grouped into 160 events and presented extremely promising results. Noise was perfectly rejected and the NN was able to identify 161 patterns out of the 160, with a single false-positive. Thus on simulated data, the authors showed the ability of the NN to detect young showers with very low error rates \cite{szadkowski2015artificial}. While Szadkowski et al. (2014) presented the first known implementation of NNs on neutrino data, the paper faced a few shortcomings. First, the paper did not indicate the rationale behind the  selection of various hyperparameters, network architecture and activation functions. Next, the paper did not mention the nature of the test cases to indicate performance of the network across varied test data.  Finally, only error rates were discussed as a metric however, precision-recall (PR) and Receiver Operating Characteristics (ROC) curves could have proved a better measure of network performance in regard to the classification task it performed. 

\section{Convolutional Neural Networks (CNNs)}
Detectors are often built to produce high-resolution images of particle movements and interactions. Acciarri et al. (2016) used CNN architectures to reconstruct neutrino scattering interactions in such images. They explored the use of CNNs for detector images that were very information sparse and often empty. 22,000 events per type of particle were used for training the CNN in batches. Both high and low-resolution images were provided as separate demonstrations to mimic realistic scenarios. The authors combined two CNN architectures - Faster-RCNN for particle detection, followed by AlexNet for particle classification \cite{acciarri2016long}. Results can be deemed quite promising as the authors noted the combined network's ability to distinguish track-like particles from shower particles very well. For high-resolution images, track-like particles had \textit{87.2\%} accuracy and shower-like particles had \textit{81.3\%} accuracy. For low-resolution images, the score was lower with \textit{85.8\%} accuracy for track-like images and \textit{77.3\%} for shower-like images \cite{acciarri2016long}. Based on the accuracy scores, it was concluded that there was reasonable localisation of both shower and track particles for high and low resolution images. However, the authors did not report precision-recall (PR) scores or the area under ROC curves that are imperative in classification scenarios. 
 
Neutrino event classification experiments often also involve tagging and identification of on-beam event images for a neutrino interaction.  Acciarri et al. (2016) additionally developed a methodology that identified neutrino interactions on single-plane images and cropped them around the interaction region. They then applied the network described in their previous work to classify particles in the cropped images \cite{acciarri2017convolutional}. Acciarri et al. (2016) defined two classes for the classification task - Monte Carlo neutrino events and purely cosmic events and trained with InceptionResNet \cite{acciarri2017convolutional}. The authors reported an \textit{80\%} accuracy score during training but reported lower validation scores of \textit{78\%}. Acciarri et al. (2016) additionally reported performance via selection efficiency for neutrino events to be a positive \textit{80.1\%}. They believed that this efficiency would improve if all three planes were used instead of a single plane \cite{acciarri2017convolutional}. 
 
Acciarri et al. (2017) continued on their study from \cite{acciarri2017convolutional} by extending their work from single-plane images to three-planes and combining it with optical detector data \cite{collaboration2016convolutional}. High definition input images of simulated neutrino images and cosmic images (as background) were used. A new truncated network based on ResNet was designed. The authors discussed the compromise of having fewer layers learning fewer filters, but preserving resolution, allowing for exposure to detailed features. Distribution of neutrino classification scores showed a very good separation between the two types of events. The selection efficiency improved to \textit{85\%} (from \textit{80.1\%} with one plane images) \cite{collaboration2016convolutional}.

Image based detection and classification of neutrino particles were examined further through means of different representations of data and architectures. Adams et al. (2018) continued on the work of Acciari et al. (2016, 2017) by developing a Convolutional Neural Network that could predict objects in image data at the pixel level \cite{collaboration2019deep}. Adams et al. (2018) trained U-ResNet, a deep semantic segmentation network via supervised learning. First, they used transfer learning techniques by training the first half of the network on the dataset from a previous work that contained single particle images \cite{acciarri2017convolutional}. Then, they developed a new loss factor called pixel-wise loss (PL) weighing factor. This factor was multiplied by a single pixel's loss contribution to the total loss of an image. Thus, complex sections of the image obtained a higher weighted pixel loss, allowing the network to focus it's training on those regions. The process was monitored using the Incorrectly Classified Pixel Fraction (ICPF) metric. The ICPF indicates the average value of incorrectly classified pixel per image over all images on all events in a sample \cite{zhang2008new}. The network was trained on 100,000 images and then tested on 20,000 images. U-ResNet achieved an average ICPF of \textit{6.0} for electron neutrinos and \textit{3.9} for muon neutrinos. They noted that U-ResNet could classify pixels from low energy and simple topologies fairly well with mean ICPF scores of \textit{2.3} and \textit{3.9} respectively. The authors additionally obtained real detector data and compared the results of their network with those obtained by physicists. However, physics methodology had a better, lower mean ICPF score of \textit{1.8} for the electron samples while the network scored a mean ICPF of \textit{2.6} \cite{collaboration2019deep}. The authors justified this difference in performance to be due to lack of specialised physics determined features. Despite this, the study lays foundation for a new methodology for training and examining data from a new pixel level perspective.

Aurisano et al (2016) developed a technique called Convolutional Visual Network (CVN) based on CNNs to reconstruct neutrino energy and flavour. The authors designed the network to have two distinct views of the same image, rather than representing a single image in multiple colour channels. The network was trained using mini-batches on 3.7 million simulated neutrino events and tested on 1 million samples \cite{aurisano2016convolutional}. To measure CVN's performance, it was first compared against existing metrics. Measurement-optimised efficiency scores were obtained from existing physics metrics and compared against that of the CVN. CVN scored an efficiency of \textit{58\%} versus the existing \textit{57\%} efficiency for muon neutrino interactions. The authors felt that while the improvement was modest, it was still in the positive direction. CVN however scored \textit{40\%} efficiency over the pre-existing metric of \textit{35\%} for electron neutrinos. Additionally, the authors computed a Figure of Merit (FOM) to assess the performance of signal identification over background noise for oscillation parameters. Overall, the CVN obtained a range of efficiency scores from \textit{17.4\%} at the lowest to \textit{66.4\%} at the highest for various parameters. The authors found the results quite promising since they performed minimal event reconstruction and found positive performance with a single algorithm \cite{aurisano2016convolutional}. Moreover, the CVN developed was used on atypical images, specifically the readout of a calorimeter. This study therefore opened up the possibility of using a different form data which might be extendable to other detectors as well

\section{Graph Neural Networks (GNNs)}
IceCube is a neutrino observatory at the South Pole that searches for high energy neutrino events \cite{icecube2013evidence}. It observes two classes of such events - neutrino interactions within the detector and high energy cosmic interactions in the upper atmosphere \cite{icecube2013evidence}. Choma et al.(2018) in their study on data from IceCube proposed that the irregular geometry of the detectors could be modelled as a graph with vertices as sensors and edges as learned functions of the sensors' spatial coordinates. They stated large asymmetry between positive and negative events to be the main challenge. The authors proposed the use of Graph Neural Networks (GNNs) for this work. The authors considered muon neutrinos as positive signals and the rest as background noise. As the background was much larger in terms of magnitude than the signal, a high rejection power was required. The GNN was initialised as a fixed, weighted, directed graph. 25,250 events were generated as signal and 109,491 events were generated as background. The performance of the classification algorithm was noted against physics results and CNN scores \cite{choma2018graph}. As per Choma et al. (2018), physics-derived metrics reported \textit{0.987} signal to noise ratio for events per year and CNN reported \textit{0.937} signal-to-noise ratio. The GNNs showed clearly superior results by reporting \textit{2.980} signal-to-noise ratio for events per year. The GNN outperformed physics metrics by identifying three times more positive events \cite{choma2018graph}.


Based on studies discussed, many Neural Networks trained for neutrino detection are variations of CNNs \cite{baldi2014searching, szadkowski2015artificial, edelen2016neural}. Most of the problems discussed involved identification of particles from background noise and then classification, often in the form of images \cite{acciarri2016long, acciarri2017convolutional}. However, typical CNN architectures requires conversion of data to images, resulting in significant volumes of data \cite{qi2017pointnet}. Further, in trying to quantize data, several unwanted features (artefacts) may render on the images distorting results \cite{qi2017pointnet, qi2017pointnet++}. In order to avoid such pitfalls, this thesis makes use of PointNet, a network that directly consumes point clouds without requiring conversion to a regular input like 3D voxels \cite{qi2017pointnet}. Due to the varied nature of the datasets and questions being answered, different metrics got reported. None of the studies reported Precision-Recall (PR) scores that are useful to understand the learning abilities of classifiers \cite{goodfellow2016convolutional}. This thesis instead focuses on measuring recall, especially for the positive (events) class and employs a combination of metrics to ensure that any effects of class imbalance are mitigated. 

% Very few studies stated comparisons of their networks against the standard physics results, making it hard to draw conclusions. Often, the training data used was simulated and left questions regarding its extensibility to real-world data \cite{collaboration2016convolutional, collaboration2019deep, acciarri2016long, acciarri2017convolutional}. 
