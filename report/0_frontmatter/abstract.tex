\begin{abstracts}  

Particle physics involves examination of sub-atomic particles and their interactions. The main challenge in this field often lies in the separation of background noise from event signals. Most neural networks in the field use CNNs for particle classification. But this often leads to loss of information when converting data to images. This thesis examines the application of PointNet - a 3D classification network for KM3NeT neutrino data. The thesis has a two-fold interest. First, it wishes to investigate the role of 3D deep learning in neutrino identification. Next, it wishes to apply the network on KM3NeT data to save neutrino information while discarding background noise. The data is split into three datasets and trained individually. Feature engineering is performed and the resulting point clouds are converted to 3D meshes. A majority voting ensemble technique is used to combine predictions from the three models. The network showed promising results with a 95\% recall for the positive class and perfect precision. The model also demonstrated perfect recall for the noise class. Being the first known work of its kind, results from the thesis indicate PointNet to be a viable methodology for future neutrino research.

{\bf Keywords:} \textit{PointNet}, \textit{neutrino detection}, \textit{classification}, \textit{3D deep learning}, \textit{KM3NeT}
\end{abstracts}


